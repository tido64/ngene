\section{Conclusion}
Ngene Development Framework was designed to enable the user to shift its focus from the technical details of implementation to the model itself.

It was shown that it is entirely possible to implement two completely different models using the same code base provided by the framework. It was also shown that the models re-implemented into this framework, were able to perform at least as well as the original implementation did. The numbers and results we've seen, back these claims up.

We've also been able to perform a more thorough comparison of the two engines without having to a


\subsection{Further Work}

Regretfully, some experiments were not conducted that would have showed off the comparison part of the framework.

\subsubsection{Improvements to Ngene}
\label{sec:improvements}
There are some aspects of the engine that I would have changed or improved but did not have the time to do so.

\begin{itemize}
	\item\textbf{Further optimizations}
		\begin{itemize}
			\item Currently, the messages are gathered by looping through the cells and finding out what their neighbours are. This is a $O(n^{2})$ algorithm and can be much improved. One way of doing it is to take a snapshot of the organism and pass that object to the cells. The benefit of doing it this way is that it is fast (in the order of $O(n)$) and it also opens up the possibility to simulate chemical diffusion across more cells than the immediate neighbours of a cell.

			\item The use of \texttt{<map>} comes with a penalty as mentioned in chapter~\ref{tbl:speed}. A three-dimensional implementation of a \texttt{<vector>} should be considered as it would give constant lookup time and negate the need to search for nearby elements.

			\item The Mersenne twister is often criticized for not being very elegant. There has been suggestions that point to other implementations of a pseudo-random number generator that outperform the Mersenne twister while maintaining equal or better quality of randomness. There is also a new implementation of Mersenne twister that should be considered. It is much faster and makes use of SIMD instructions in recent CPUs. Genetic algorithms in general are very dependent on a good random number generator, and having a fast one is only beneficial to the whole engine.

			\item Ngene has implemented OpenMP\footnote{http://www.openmp.org/} in order to utilize multi-core CPUs. This feature was, however, disabled because the framework is not thread-safe. As multi-core CPUs are becoming more common, the framework should be able to utilize such facilities.
		\end{itemize}

	\item\textbf{Making the core interchangeable}

	The genetic algorithm itself should be implemented as a separate module, making it possible to change algorithm depending on the experiments conducted.

	\item\textbf{Ability to swap modules mid-experiment}

	The possibility to swap modules based on circumstances in a population, for instance, half way through an evolution or when the population is converging may produce interesting results. This can be used to simulate genetic drift or other natural phenomena.
\end{itemize}

\subsubsection{Other models}

TODO
