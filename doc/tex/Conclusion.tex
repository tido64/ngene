\section{Conclusion}

\subsection{Further Work}

\subsubsection{Improvements to Ngene}
\label{sec:improvements}
There are some aspects of the engine that I would have changed or improved but did not have the time to do so.

\begin{itemize}
	\item\textbf{Improving the message interface of Ngene Development Framework}

	Currently, the messages are gathered by looping through the cells and finding out what their neighbours are. This is a $O(n^{2})$ algorithm and can be much improved. One way of doing it is to take a snapshot of the organism and pass that object to the cells. The benefit of doing it this way is that is fast (in the order of $O(n)$) and it also opens up the possibility to implement chemical diffusion across more cells than the immediate neighbours.

	\item\textbf{Further optimizations}

	The use of \texttt{<map>} comes with a penalty as mentioned in chapter~\ref{tbl:speed}. A three-dimensional implementation of a \texttt{vector} should be considered as it would give constant lookup time and negate the need to search for nearby elements.

	The Mersenne twister is often criticized for not being very elegant. There has been suggestions that point to other implementations of a pseudo-random number generator that outperform the Mersenne twister while maintaining equal or better quality of randomness. Genetic algorithms in general are very dependent on a good random number generator, and having a fast one is only beneficial to the whole engine.

	Ngene has implemented OpenMP\footnote{http://www.openmp.org/} in order to utilize multi-core CPUs. This feature was, however, disabled because the framework is not thread-safe. As multi-core CPUs are becoming more common, the framework should be able to use it to its own advantage.

	\item\textbf{Making the core interchangeable}

	The genetic algorithm itself should be implemented as a separate module, making it possible to change algorithm at any time.

	\item\textbf{Ability to swap modules mid-experiment}

	The possibility to swap modules based on circumstances in a population, for instance, half way through an evolution or when the population is converging.
\end{itemize}

\subsubsection{Other models}

TODO

