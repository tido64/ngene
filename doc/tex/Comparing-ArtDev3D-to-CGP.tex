\section{Comparing ArtDev3D to CGP}
We've now seen that implementations of the models using the framework, were able to reproduce the results achieved in experiments conducted by their inventors. However, there is one other purpose of this framework that we still need to discuss. That is the ability to compare two models with a minimum set of interfering factors.

\subsection{Design}
The design of CGP is very elegant. The initialization stage does nothing more than implementing the digital circuit, or cell program, from the genotype. And the cell program just feeds input into this program. Output is parsed, and the cell is updated. Very simple. ArtDev3D is deeply rooted in biology and this is reflected in the code. Even with the new implementation, the code is much more complex compared to CGP.

\subsection{Stability}
From the results we've seen so far, there are a number of things that can be said about these models. Most prominently, we see that ArtDev3D is able to evolve perfect specimens within relatively short time. With the sphere experiment, perfect fitness was achieved within just 500 generations. What can't be seen from the results is that it would achieve this typically within 50 generations. In contrast, the CGP model requires a lot of generations in order to reach higher fitness. At 10 000 generations, fitness rarely exceeded 0.9. Even when doubling that number, the population was still improving. Obviously, one cannot set it to run forever. The problem seemed to lie in the development method used in the models.

In ArtDev3D, the development or growth would stop after a certain amount of steps regardless of how long the development time. This ensures that the organisms will grow to target shape without overgrowing. H{\o}ye\cite{hoye2006} demonstrated this property in his thesis, and this is reflected by the model's ability to achieve high fitness fast. From the results obtained, it seemed that it is difficult to correctly evolve desired phenotype without knowing how many development steps will be needed. For instance, with the French flag experiment, using four development steps was optimal compared to three or five. Three was obviously not enough because the cells do not have the time to grow properly, while five was too much because the cells started to ``deform''. The new test showed this tendency as well, but it is difficult to state anything for certain with these results. As I've mentioned, one will have to develop a tool to look into each development step ascertain this behaviour.
